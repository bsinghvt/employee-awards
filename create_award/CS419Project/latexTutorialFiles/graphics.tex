% This latex file can be processed with either latex (plus dvips) or pdflatex,
% provided the files cat.eps and cat.pdf reside in the same folder as this file.

\documentclass[11pt]{article}          % article class
\usepackage{graphicx}                  % provides \includegraphics command
\usepackage[colorlinks,urlcolor=blue]{hyperref} % for live links in pdf output
\topmargin -.7in \textheight 9.35in
\oddsidemargin 0in \textwidth 6.5in

%%%%%%%%%%%%%%%%%%%%%%% How to use the fancyhdr package %%%%%%%%%%%%%%%%%%%%
\usepackage{fancyhdr}                % use fancyhdr to customize hdrs & ftrs
\renewcommand{\headrulewidth}{0pt}   % suppress rule across top(fancyhdr default)
\headheight=13.6pt                   % make enough room for header
% The lines below set values for left, right, center for header & footer:
\lhead{\textit{Including EPS Graphics in \LaTeX}}  % left head
\chead{}                             % empty center head
\rhead{\thepage}                     % page number on right head
\lfoot{Academic \& Research Computing, RPI}  % text for left foot
\cfoot{}                             % empty center foot              
\rfoot{\today}                       % today's date on right foot
\pagestyle{fancy}                    % use fancyhdr style
%%%%%%%%%%%%%%%%%%%%%%%%%%%%%%%%%%%%%%%%%%%%%%%%%%%%%%%%%%%%%%%%%%%%%%%%%%%%%%

\title{\vspace*{-.9in}\textbf{Including Graphics in \LaTeX}\vspace{-12pt}}
% Use substitute author & date arguments to get desired text and spacing:
\author{\textit{To get the most out of this document, also view the file
  \texttt{graphics.tex}}\\
\textit{to see the commands that included the graphics}}
\date{\vspace{-30pt}} 

\begin{document}                       % Begin document text
\maketitle                             % prints title information

\section*{The graphicx package and the figure environment}
\LaTeX\ provides a graphics package that allows you to include in
your finished document graphics files which you have prepared using
a graphics program such as Xfig, CorelDraw, MayuraDraw, Gnuplot,
Maple, Matlab, etc.  It's also possible to create an eps
(Encapsulated PostScript) file from any Windows application by
printing to a file, as described in the
section below. A general rule of thumb for graphics formats: use pdf
or eps for vector graphics (line art, graphs, etc), jpg for
photographs, and png for screen shots.

When you use \LaTeX\ plus \texttt{dvips},% to process your document,
your graphics files must be in eps format.  pdf\LaTeX\ (which
produces a pdf file directly) accepts pdf, jpg, and png,
but \textit{not} eps.  If you want to use pdf\LaTeX\ and your
graphics files are in eps format, you can convert them to pdf using
the \texttt{epstopdf} utility, which is most likely on your system.
There is also a \texttt{jpeg2ps} utility for converting jpg files to
eps.

For illustrating how to use the \textbf{graphicx} package to
include graphics, we will use a small line drawing of a cat that exists
in both eps and pdf formats (called \texttt{cat.eps} and \texttt{cat.pdf}).  
These files are in the same folder as this file (\texttt{graphics.tex}), 
which means \LaTeX\ and pdf\LaTeX\ can find them. (It's easiest to
keep your \texttt{.eps} and/or \texttt{.pdf} files 
in the same folder as the \LaTeX\ file that uses them. \LaTeX\ looks for files
first in the current directory and then in its normal input path.)  
Below we use the \verb+\includegraphics+ command from the 
\textbf{graphicx} package to insert the picture of the cat right
here:
\includegraphics{cat}  % just plop the graphic right here.
Note that if you use the command \verb+\includegraphics{cat}+, omitting the
filename extension, \LaTeX\ will find \texttt{cat.eps} and
pdf\LaTeX\ will find \texttt{cat.pdf}.  
 
For a more sophisticated treatment, we can put the \verb+\includegraphics+ 
command inside the \texttt{figure} environment, which will float the graphic 
to be sure there is room for it. (See Figure~\ref{fig:cat}.) 
% Run latex twice to resolve this reference.
LaTeX usually places it at the top or bottom of the current page or
at the top of the following page. In this example, we'll center,
scale and rotate the graphic. Look for it at the bottom of this page.
The figure environment also provides a ``caption" command.
 
\begin{figure}[b]  % "b" specifies that it goes at the bottom of page
\centering         % declaration corresponding to the center environment
\includegraphics[scale=2,angle=90]{cat}
\caption{A bigger pussycat, on its side}
\label{fig:cat}    % \label the figure to be able to \ref it in the text
\end{figure}

\section*{How to make EPS files from Windows applications}
Even if the application you are using to create your figure cannot
export as EPS, you can still create an EPS file from a Windows
application (such as Word or Excel) as long as you have a
(non-Microsoft) PostScript printer driver installed on the PC.  The
plain, generic Apple LaserWriter 16/600 PS is good for black \& white 
graphics. If you don't have such a PS printer on your PC, use the 
Add Printer feature to add it; do \textbf{\textit{not}} use a Microsoft,
HP, or Tektronix driver. If you need color, use the Generic Adobe PS
printer; you will probably need to download this from the Adobe web site
(\url{http://www.adobe.com/support/downloads/detail.jsp?ftpID=1500})
before using Add Printer. Hint: When adding this printer, select that
it's a local printer connected to your computer. Then, select
``FILE local port" from the list of ports.

\subsection*{Step 1: Printing to a file}
From your PC application, select Print. When the Print window appears, 
choose to print to a file, and select the generic PS printer from your
list of printers. Then be sure to select the EPS format from among the 
printer options, as follows:\\ 
On WinXP, click Properties, then click the ``Advanced...''
button.  In the next window, click the ``+'' next to PostScript Options, 
then change PostScript Output Options to ``Encapsulated PostScript (EPS)".

You will be asked to supply a file name. Windows will insist on
giving the name a .prn extension, so afterwards you must rename the
file with a .ps extension (e.g., mygraphic.ps). (NOTE: if you type
quotes (``...") around your file name, Windows will not add the .prn
extension and you can avoid the renaming step.)

After making a minor correction (Step 2 below), during which you can
give your file a .eps extension, you should be able
to import this new eps file into your LaTeX file.

\begin{quote}\small
\textbf{PowerPoint Note:} The EPS produced from PowerPoint is especially 
nasty, containing some illegal code, and therefore requires an extra step
to clean it up before doing  Step 2. (Also note that, when creating your 
graphic in PowerPoint, it's best to use a portrait page with no background.)
You can clean up the PowerPoint PS file as follows:\\[2pt]
- Using GSView, open the .ps file you just created.\\
- From the File menu, select ``Convert...''\\
- In the Device box at the left, select ``epswrite'', 
and for Resolution, select 300dpi. Click OK.\\[2pt]
(When you are asked for a file name, it's easiest to choose a name with
the extension .ps, saving the .eps extension for the final step, below).
\end{quote}
 
\subsection*{Step 2: Correcting the Bounding Box}
Windows printer drivers usually create the EPS graphic with a
``Bounding Box" (an invisible box that determines the boundaries of
the graphic) the size of a full page, even if the graphic itself is
small. To change the BB to include only the graphic, open the .ps
file in GSView. To display the BB, pull down the Options menu and
click on ``Show Bounding Box." To correct the BB, from the File
menu, select ``PS to EPS". In the window that appears, check
``Automatically calculate Bounding Box" and click ``Yes". You will
then be asked to provide a name for a new file with the correct BB;
this time give it the .eps extension.
You can use the File menu to open the new file and view the
corrected BB. Your new file should import correctly into LaTeX using
the \verb+\includegraphics+ command as described above.  

\section*{For more information}
In Memo RPI.109, \href{http://www.rpi.edu/campus/doc/acs.memos/rpi109.pdf}
{Text Formatting with \LaTeX}, you can find more
on the figure environment in section 4.12, and more information on
including graphics in section 7.4.\medskip

The RPI example file \texttt{exrotating.pdf} and it�s corresponding \LaTeX\ file 
\texttt{exrotating.tex} show how to
include landscape figures and tables:\\[2pt]
\url{http://www.rpi.edu/dept/arc/training/latex/Examples/exrotating.pdf}\\
\url{http://www.rpi.edu/dept/arc/training/latex/Examples/exrotating.tex}\medskip

Official documentation for the \LaTeX's graphics bundle is in \texttt{grfguide.pdf}; 
look for it on your system. 
The information on the \texttt{graphicx} package is in Section 4.4.\medskip

For an exhaustive treatment of including graphics in LaTeX documents see
the excellent document, ``Using Imported Graphics in \LaTeX 2e." You can
find it at\\ \url{http://www.ctan.org/tex-archive/info/epslatex.pdf}. 

\end{document}